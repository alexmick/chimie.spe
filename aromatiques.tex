\setatomsep{2em}
\setarrowdefault{0,1.5,black}
\setcrambond{5pt}{}{}
\chapter{Hydrocarbures aromatiques}
\section{Halog�nation}
\noindent $\star$ \textbf{Bilan:} Ph - H + X$_2 \longrightarrow$ Ph - X + H$\,$X  \\
$\star$ \textbf{Catalyseurs} selon la nature de X$_2$:
\begin{description}
	\item[Br:] FeBr$_3$ g�n�r� \textit{in situ} par action de Br$_2$ sur Fe ( 2$\:$Fe + 3$\:$Br$_2 \longrightarrow\ 2\:$FeBr$_3$)
	\item[Cl:] AlCL$_3$ ou FeCl$_3$
	\item[I:] Trop mauvais rendement
	\item[F:] Trop explosif
\end{description}
$\star$ \textbf{M�canisme}:
\schemestart
\chemfig{\lewis{246,Br}-\lewis{206,Br}} \+ \chemfig{\lewis{4|,Fe}-Br_3}
\arrow{<=>}
\chemfig{[0,1.5]\lewis{357,Br}-[1]\lewis{13,\chemabove{Br}{\scriptstyle\oplus}}-[7]\chemabove{Fe}{\scriptstyle\ominus}Br_3}
%\arrow{<=>}
%\chemleft(\chemfig{Br^\oplus},\chemabove{Fe}{\scriptstyle\ominus}Br_3\chemright)
\schemestop
\vspace{1cm}
\begin{center}
	\schemestart
	\chemfig{*6(-=-(-H)=[@{dbl}]-=)} \+ \chemfig{[0,1.5]@{br}\lewis{357,Br}-[@{li}1]@{br2}\lewis{13,\chemabove{Br}{\scriptstyle\oplus}}-[7]\chemabove{Fe}{\scriptstyle\ominus}Br_3}
	\chemmove{\draw[->,shorten <=1pt,shorten >=5pt](dbl).. controls+(45:2cm)and+(90:1cm).. (br);
	\draw[->,shorten <=2pt,shorten >=1pt](li).. controls+(-45:0.5cm)and+(-90:0.5cm).. (br2);
	}
	\arrow
	\chemfig{*6(-=-(<|[:50]@{h}H)(<:[:-15]Br)-(-[,-0.2,,,draw=none]\scriptstyle\oplus)(-[,0.01,,,draw=none]\lewis{2|,})-=)} \+ \chemfig{Br-[@b]\chemabove{Fe}{\scriptstyle\ominus}Br_3}
	\chemmove{\draw[->,shorten <=2pt,shorten >=1pt](b).. controls+(90:1cm)and+(0:1cm).. (h);}
	\arrow[-90,1.5]
	\chemfig{*6(-=-(-Br)=-=)} \+ HBr \+ FeBr$_3$
	\schemestop
\end{center}
\newpage
\section{Alkylation}
\noindent $\star$ \textbf{Bilan:} Ph - H + R - X $\longrightarrow$ Ph - R + H$\,$X  \\
$\star$ \textbf{M�canisme}:
\begin{center}
\schemestart
	\chemfig{H_3C-[,0.7]{(}CH_2{)}_{2}-[,1.5]@{cl}\lewis{206,Cl}} \+ \chemfig{@{al}\lewis{4|,Al}-Cl_3}
	\chemmove{\draw[->,shorten <=2pt,shorten >=5pt](cl).. controls +(90:1cm) and +(180:0.8cm) ..(al);}
	\arrow{<=>}
	\chemfig{H_3C-[,0.7]{(}CH_2{)}_{2}-[,1.2]\lewis{26,Cl^\oplus}-\chemabove{Al}{\scriptstyle\ominus}Cl_3}
\schemestop \vspace{5mm}
\schemestart
	\hspace{5.9cm} %pour aligner les fl�ches
	\arrow{<=>}
	\chemleft(\chemfig{H_3C-CH_2-\chemabove{\lewis{6|,CH_2}}{\scriptstyle\oplus}}, \chemabove{Al}{\scriptstyle\ominus}Cl$_4$\chemright)
\schemestop \vspace{5mm}
\schemestart
	\hspace{5.9cm} %pour aligner les fl�ches
	\arrow{<=>[\scriptsize Transposition][]}
	\chemleft(\chemfig{H_3C-\chemabove{\lewis{6|,CH}}{\scriptstyle\oplus}-CH_3}, \chemabove{Al}{\scriptstyle\ominus}Cl$_4$\chemright)
\schemestop \vspace{5mm}
\schemestart
	\hspace{5.5cm} %pour aligner les fl�ches
	\arrow{<=>}
	\chemfig{CH(-[:135]H_3C)(-[:225]H_3C)-\lewis{26,Cl^\oplus}-\chemabove{Al}{\scriptstyle\ominus}Cl_3}
\schemestop
\end{center}
\begin{center}
	\schemestart[,0.8]
	Et donc:
	\chemfig{[:-30,0.7]*6(=-=(-H)-=-)} \+ \chemfig{-[:30]-[:-30]-[:30]Cl}
	\arrow(debut.mid east--.south west)[45,0.7]
	\chemfig{[:-30,0.7]*6(-=-(--[:-30]-[:30])=-=)}
	\arrow(@debut.mid east--.north west)[-45,0.7]
	\chemfig{[:-30,0.7]*6(-=-(-(-[:-45])-[:45])=-=)}
	\schemestop
\end{center}
\section{Acylation}
\noindent $\star$ \textbf{Bilan:} Ph - H + \chemfig{C(-[:135]H_3C)(-[:225]Cl)=O} $\longrightarrow$ \chemfig{Ph-C(=[:90,0.7]O)-CH_3} + HCl$_{(g)}$ \\
$\star$ \textbf{Catalyse} par AlCl$_3$:
\begin{center}
	\schemestart
		\chemfig{C(-[:135]H_3C)(-[:225]\lewis{357,Cl})=\lewis{17,O}} \+ \lewis{4|,AlCl_3}
		\arrow{<=>}
		\chemfig{C(-[:135]H_3C)(-[:225]\lewis{35,Cl^\oplus}-[:-20]\chemabove{Al}{\scriptstyle\ominus}Cl_3)=\lewis{17,O}}
		\schemestop \vspace{5mm}
		\schemestart
		\arrow{<=>}[,0.9]
		\chemleft(
		\subscheme{
			\chemleft\{
			\subscheme{
				\chemfig{H_3C-\lewis{6|,\chemabove{C}{\scriptstyle\oplus}}=\lewis{17,O}}
				\arrow{<->[Cation Acylium][]}[,1.7]
				\chemfig{H_3C-C~\lewis{0,\chemabove{O}{\scriptstyle\oplus}}}
				}
			\chemright\}
			, \chemfig{Cl-\chemabove{Al}{\scriptstyle\ominus}Cl_3}
			}
		\chemright)
	\schemestop
\end{center}
$\star$ \textbf{M�canisme}:
\begin{center}
	\schemestart
		\chemfig{C(-[:135]H_3C)(-[:225]\lewis{35,Cl^\oplus}-[:-20]\chemabove{Al}{\scriptstyle\ominus}Cl_3)=\lewis{17,O}}
		\+
		\chemfig{[:-30,0.7]*6(=-=(-H)-=-)}
		\arrow[,0.9]
		\chemfig{[:-30,0.7]*6(=-(-[,-0.2,,,draw=none]\scriptstyle\oplus)(-[,0.01,,,draw=none]\lewis{7|,})-(-[:45]H)(-[:-45](=[:-90]O)-[:45])-=-)}
		\arrow{0}[,0] \+ \chemfig{Cl-\chemabove{Al}{\scriptstyle\ominus}Cl_3}
		\arrow[-90,0.7]
		\chemfig{[:-30,0.7]*6(-=-(-(-[:45])=[:-45]O)=-=)}
		\+ HCl$_{(g)}$ \+ AlCL$_3$
	\schemestop
\end{center}
Mais AlCl$_3$ r�agit sur la c�tone aromatique, on fait donc une hydrolyse acide.
\begin{center}
	\schemestart
	\chemfig{[:-30,0.7]*6(=-=(-(-[:-45])=[:45]\lewis{3,\chemabove{O}{\ \scriptstyle\oplus}}-[,1.1]\chemabove{Al}{\scriptstyle\ominus}Cl_3)-=-)}
	\arrow{->[H$_2$O][H$^\oplus$]}
	\chemfig{[:-30,0.7]*6(=-=(-(-[:-45])=[:45]\lewis{20,O})-=-)}
	\+ Al$^{3\oplus}$ \+ 3Cl$^\ominus$
	\schemestop
\end{center}
$\star$ \textbf{Application aux anhydrides}
\begin{center}
	\schemestart
		\chemfig{(-[:210])(=[:90]O)-[:-30]@{dnl}\lewis{57,O}-[:30](=[@{dl}:90]O)-[:-30]} \+ \chemfig{@{al}\lewis{4|,Al}-Cl_3}
		\chemmove[->,shorten >=4pt]{
			\draw[blue, shorten <=4pt ](dnl).. controls+(-45:1cm)and+(-120:1cm).. (al);
			\draw[red, shorten <=2pt](dl).. controls+(0:1cm)and+(120:1cm).. (al);
		}
		\arrow(start.mid east--.south west){<=>}[45,0.8,red]
		\chemfig{(-[:210])(=[:90]O)-[@{a1}:-30]\lewis{57,O}-[@{a2}:30](-[@{b1}:90]@{b2}\chemabove{O}{\scriptstyle\oplus}-\chembelow{Al}{\scriptstyle\ominus}Cl_3)-[:-30]}
		\chemmove[->]{
			\draw[shorten >=2pt](a1).. controls +(225:2cm) and +(-45:2cm) ..(a2);
			\draw(b1).. controls +(180:1cm) and +(180:1cm) ..(b2);
		}
		\arrow(--un){<=>}[,0.8]
		\chemfig{(-[:-140])(=[:90]O)-[:-25,0.2,,,draw=none]\scriptstyle\oplus}
		\+ \chemfig{(=[:210]\lewis{46,O})(-[:-30])-[:90]\lewis{24,O}-\chembelow{Al}{\scriptstyle\ominus}Cl_3}
		\arrow(@start.mid east--.north west){<=>}[-45,0.8,blue]
		\chemfig{(-[:210])(=[:90]O)-[:-30]O(-[:90,0.4,,,draw=none]\scriptstyle\oplus)(-[:-90]\chembelow{Al}{\scriptstyle\ominus}Cl_3)-[:30](=[:90]O)-[:-30]}
		\arrow(--deux){<=>}[,0.8]
		\chemfig{(-[:-140])(=[:90]O)-[:-25,0.2,,,draw=none]\scriptstyle\oplus}
		\+ \chemfig{(=[:90]\lewis{13,O})(-[:-30])-[:210]\lewis{35,O}-[:-30]\chembelow{Al}{\scriptstyle\ominus}Cl_3}
		\arrow(@un--@deux){<->[][*{0.180}\scriptsize identiques]}
	\schemestop
\end{center}
Et ensuite
\begin{center}
	\schemestart
		\chemfig{[:-30,0.7]*6(=-(-[,-0.2,,,draw=none]\scriptstyle\oplus)(-[,0.01,,,draw=none]\lewis{7|,})-(-[:45]@{h}H)(-[:-45](=[:-90]O)-[:45])-=-)}
		\+ \chemfig{Cl_2\chemabove{Al}{\scriptstyle\ominus}(-[:-90]Cl)-[,1.2]O-(-[:-30])=[:30]O}
		\arrow(.base east--.base west)[,0.8]
		\chemfig{[:-30,0.7]*6(-=-(-(-[:45])=[:-45]O)=-=)}
		\+ HCl$_{(g)}$ \+ \chemfig{(-[:210])(=[:90]O)-[:-30]O-[:30]AlCl_2}
	\schemestop
\end{center}
\section{Nitration}
\noindent $\star$ \textbf{Bilan:} Ph - H + HNO$_3 \longrightarrow$ Ph - NO$_2$ (nitrobenz�ne) + H$2$O  \\
Dans un \textbf{m�lange sulfonitrique} (H$_2$SO$_4$ \& HNO$_3$ concentr�s) ou de l'acide nitrique concentr� (fumant) \\
$\star$ \textbf{M�canisme}:
	HNO$_3$ + 2 H$_2$SO$_4$ $\longrightarrow \underbrace{\mathrm{NO}_2^\oplus}_{\mathrm{Cation\ Nitronium}}$ + H$_3$O$^\oplus$ +2 HSO$_4^\ominus$
\begin{center}
	\schemestart
	\chemfig{[:-30,0.7]*6(=-=(-H)-=-)} \+ \chemfig{\lewis{35,O}=\chemabove{N}{\scriptstyle\oplus}=\lewis{17,O}}
	\arrow[,0.7]
	\chemfig{[:-30,0.7]*6(=-(-[,-0.2,,,draw=none]\scriptstyle\oplus)(-[,0.01,,,draw=none]\lewis{7|,})-[@{a}](-[@{c}:45]@{h}H)(-[:-45]\chemabove{N}{\scriptstyle\oplus}(=[:-90]\lewis{57,O})-[,0.9]\lewis{026,O}-[:48,0.5,,,draw=none]\scriptstyle\ominus)-=-)}
	\arrow{->[\chemfig{[,0.7]@{b}\lewis{246,O}^\ominus-[,0.8]S(=[:90]O)(=[:-90]O)-OH}][]}
	\chemfig{[:-30,0.7]*6(=-=(-\chemabove{N}{\scriptstyle\oplus}(=[:45,0.9]\lewis{02,O})-[:-45,0.9]\lewis{157,O}-[:10,0.5,,,draw=none]\scriptstyle\ominus)-=-)}
	\+ H$_2$SO$_4$
	\chemmove[->]{
		\draw[shorten >=1pt](c).. controls +(135:5mm) and +(135:5mm) ..(a);
		\draw[shorten <=2pt,shorten >=3pt](b).. controls +(110:1cm) and +(45:7mm) ..(h);
	}
	\schemestop
\end{center}
\section{Polysubsitutions �lectrophiles}
\subsection{R�gle de \textsc{Holleman}}
La r�gios�l�ctivit� (orto,para \textit{vs} m�ta) ne d�pend que de la nature du substituant d�j� en place. Par contre, les proportions entre ortho et para d�pendent du substrat et de l'�lectrophile.
\subsection{R�gios�lectivit�}
\begin{description}
	\item[Effet +I] $\Rightarrow$ Ortho,para-orienteur
	\item[Effet -I] $\Rightarrow$ Meta-orienteur
	\item[Effet +M] $\Rightarrow$ Ortho,para-orienteur
	\item[Effet -M] $\Rightarrow$ Meta-orienteur
\end{description}
Sachant que les effets \textbf{m�som�res} sont toujours \textbf{pr�pond�rants} sur les effets inductifs.\\
Quand il y a plusieurs substituants:
\begin{itemize}
	\item L'ordre des substitutions est crucial
	\item Les effets sont additifs
	\item Un substituant activant l'emporte toujours sur un effet d�sactivant
\end{itemize}
\section{Oxydation}
Hormis la combustion, les cycles aromatiques sont tr�s r�sistants � l'oxydation:\\
\textbf{Pas d'ozonolyse ni d'epoxydation} $\neq$ Alc�nes\\
En revanche la cha�ne substituante s'oxyde tr�s facilement, par exemple:
\begin{center}
	\schemestart
	5 \chemfig{[:-30,0.7]*6(-=-(-)=-=)} \+ 6 MnO$_4$ \+ 18 H$^\oplus$
	\arrow(.base east--.base west)[,0.6]
	5 \chemfig{[:-30,0.7]*6(-=-(-CO_2H)=-=)} \+ 6 Mn$^{2\oplus}$ \+ 14 H$_2$O
	\schemestop
\end{center}
\setatomsep{}
\setarrowdefault{}
\setcrambond{}{}{}