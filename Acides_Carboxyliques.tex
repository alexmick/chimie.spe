\setatomsep{2em}
\chapter{Acides Carboxyliques}
\section{Propri�t�s}
Structure de type $AX_3$ au voisinage du C fonctionnel. G�om�trie plane.
Pr�sence de liaisons hydrog�nes (LH) intermol�culaires $\longrightarrow$ formation de dim�res. \\ \\
\textbf{IR :}
\begin{center}
$\sigma_{\scriptsize \chemfig{C=O}}=$ 1750 � 1750 $\text{cm}^{-1}$\\
$\sigma_{\scriptsize \chemfig{O-H}}=$ 2500 � 3500 $\text{cm}^{-1}$\\ 
\end{center}
\textbf{RMN :} 
\begin{center}
H fonctionnel : tr�s d�blind�, $10<\delta<13$ ppm \\
H port� par C en $\alpha$, $2<\delta<3$ ppm \\
\end{center}
\section{Est�rification}
\begin{center}
\schemestart
\chemfig{-[::-30]-[::60](=[::60]O)-[::-60]OH}
\+
\chemfig{-\textcolor{red}{O}H}
\arrow
\chemfig{-[::-30]-[::60](=[::60]O)-[::-60]\textcolor{red}{O}-[::60]}
\+
\chemfig{H_2 O}
\schemestop
\end{center}
La vitesse augmente avec la temp�rature, mais \underline{\textbf{pas}} le rendement ! La r�action peut �tre catalys�e ($\chemfig{H_2 SO_4}$ ou $\chemfig{H_3 PO_4}$ ou APTS).
\\ \\
\paragraph{M�canisme (r�action catalys�e)}
\begin{enumerate}
\item Protonation
\item AN de l'alcool
\item R�arrangement acide/base interne
\item �limination de l'eau
\item D�protonation
\end{enumerate}
\section{D�riv�s d'acide}
\subsection{Chlorures d'acyle}
\begin{center}
\schemestart
\chemfig{-[::-30]-[::60](=[::60]O)-[::-60]OH}
\arrow{<=>[\chemfig{Cl-[::30]S(=[::60]O)-[::-60]Cl}][]}[,2]
\chemfig{-[::30]-[::-60]\lewis{2|,\chemabove{C}{\quad \oplus}}(-[::-60]OH)-[::60]O-[::-60]S(-[::80]\lewis{137,\chemabove{O}{\ominus}})(-[::-100]Cl)-[::0]Cl}
\arrow{<=>}[,1.2]
\chemfig{HO-C(-[::90]Et)(-[::-90]Cl)-O-S(-[::-60]Cl)=[::60]O}
\schemestop
\vspace{1cm}
\schemestart
\arrow[,1]
\chemfig{H-O-\lewis{0|,\chemabove{C}{\quad \oplus}}(-[::90]Et)(-[::-90]Cl)}
\+
\chemfig{SO_2}
\+
\chemfig{\chemabove{Cl}{\quad \ominus}}
\arrow[,1.2]
\chemfig{-[::-30]-[::60](=[::60]O)-[::-60]Cl}
\+
\chemfig{SO_2}
\+
\chemfig{HCl}
\schemestop
\end{center}
\subsection{Anhydrides}
\noindent Par \textbf{d�shydratation} des acides carboxyliques :
\begin{center}
\schemestart
\chemfig{12 \hspace{0.5cm} R-[::30](=[::60]O)-[::-60]OH}
\arrow{->[\chemfig{P_4 O_{10}}][-4 \chemfig{H_3 PO_4}]}[,2]
\chemfig{6 \hspace{0.5cm} R-[::30](=[::+60]O)-[::-60]O-[::60](=[::+60]O)-[::-60]R}
\schemestop
\end{center}
Par \textbf{substitution nucl�ophile} sur un chlorure d'acyle (obtention d'un anhydride \textbf{mixte}) :
\begin{center}
\schemestart
\chemfig{R_1 -[::30](=[::60]O)-[::-60]OH}
\arrow{->[\chemfig{R_2 -[::30](=[::60]O)-[::-60]Cl}][AN puis E]}[,2]
\chemfig{R_1 -[::30](=[::+60]O)-[::-60]O-[::60](=[::+60]O)-[::-60]R_2}
\schemestop
\end{center}
\vspace{2cm}
Exemple des diacides :
\begin{center}
\schemestart
\chemname{\chemfig{*6(-=(-COOH)-(-COOH)=-=)}}{Acide orthophtalique}
\arrow{->[$\Delta$][- \chemfig{H_2 O}]}[,2]
\chemname{\chemfig{*6(-=(*5(-(=O)-O-(=O)--))-=-=)}}{Anhydride phtalique}
\schemestop
\end{center}
\vspace{2cm}
\subsection{Ester}
Par substitution, m�canisme proc�dant par une addition nucl�ophile suivie d'une �limination, � partir d'un :
\paragraph{Chlorure d'acyle}
\begin{center}
\schemestart
\chemfig{Et-[::30](=[::60]\lewis{13,O})-[::-60]Cl}
\arrow{<=>[\chemfig{H-\lewis{26,O}-Me}][A.N.]}[,1.5]
\chemfig{Et-C(-[::90]\lewis{024,\chemabove{O}{\quad \ominus}})(-[::-90]Cl)-\lewis{2,\chembelow{O}{\oplus}}(-[::30]Me)-[::-30]H}
\arrow{<=>[][$-$ \chemfig{\chembelow{Cl}{\ominus}}]}[,1.5]
\chemfig{Et-[::30](=[::60]\lewis{13,O})-[::-60]\lewis{2,\chembelow{O}{\oplus}}(-[::60]Me)-[::0]H}
\schemestop
\vspace{1cm}
\schemestart
\chemfig{Et-[::30](=[::60]\lewis{13,O})-[::-60]\lewis{2,\chembelow{O}{\oplus}}(-[::60]Me)-[::0]H}
\arrow{->[\chemfig{[:-30]\lewis{4,N}*6(-=-=-=)}][a/b]}[,1.5]
\chemfig{Et-[::30](=[::60]\lewis{13,O})-[::-60]\lewis{26,O}-Me}
\+
\chemfig{\chemabove{Cl}{\ominus}-[,,,,dash pattern=on 2pt off 2pt]\chemabove{H}{\quad \oplus}N*6(-=-=-=)}
\schemestop
\end{center}
\vspace{.5cm}
\paragraph{Anhydride}
\begin{center}
\schemestart
\chemfig{Me -[::30](=[::+60]\lewis{13,O})-[::-60]\lewis{57,O}-[::60](=[::+60]\lewis{13,O})-[::-60]Me}
\arrow{<=>[\chemfig{H-\lewis{26,O}-Et}][A.N.]}[,1.5]
\chemfig{Me -[::30](=[::+60]\lewis{13,O})-[::-60]\lewis{57,O}-[::60](-[::-120]\chembelow{O}{\oplus}(-[:-150]H)(-[:-30]Et))(-[::+60]\chemabove{\lewis{024,O}}{\quad \ominus})-[::-60]Me}
\schemestop

\vspace{1cm}
\schemestart
\arrow{<=>[][]}[,1]
\chemfig{Me -[::30](=[::+60]\lewis{13,O})-[::-60]\chembelow{\lewis{157,O}}{ \ominus}}
\+
\chemfig{\chemabove{O}{\quad\oplus}(-[:120]H)(-[:-120]Et)-(=^[:60]\lewis{02,O})(-[:-60]Me)}
\arrow{->[a/b][]}[,1.5]
\chemfig{Me-[::30](=[::60]\lewis{13,O})-[::-60]\lewis{26,O}H}
\+
\chemfig{Me-[::30](=[::60]\lewis{13,O})-[::-60]\lewis{26,O}-Et}
\schemestop
\end{center}
