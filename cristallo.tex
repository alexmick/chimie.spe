\chapter{Cristallographie}
\section{Structures cristallines}
\begin{description}
	\item[Motif:] plus petite esp�ce chimique qui se r�p�te p�riodiquement.
	\item[R�seau:] ensemble des points construits par $(\vec{a},\vec{b},\vec{c})$, ce sont les lieux g�om�triques des motifs.
	\item[Noeud:] point particulier du r�seau o� se retrouve le motif.
	\item[Maille:] portion de l'espace qui par translation assure le pavage de tout le r�seau.
	\begin{itemize}
		\item La Maille �l�mentaire ne contient qu'un seul motif.
		\item La Maille Conventionnelle contient plusieurs motifs et � l'int�r�t de bien repr�senter les sym�tries dans le r�seau.
	\end{itemize}
\end{description}
%� continuer...
\section{Assemblages ioniques}
\begin{description}
	\item[Type NaCl:] Cl$^{\scriptsize\ominus}$ noeuds du r�seau, Na$^{\scriptsize\oplus}$ sites octa�driques \\
	Coordinence: [6;6] Population: $n_+=n_-=4$
	\item[Type Blende (ZnS):] S$^{2\scriptsize\ominus}$ noeuds du r�seau, Zn$^{2\scriptsize\oplus}$ la moiti� de sites t�tra�driques (par bases altern�es) \\
	Coordinence: [4;4] Population: $n_+=n_-=4$
	%� continuer...
\end{description}